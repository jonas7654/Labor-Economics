% Options for packages loaded elsewhere
\PassOptionsToPackage{unicode}{hyperref}
\PassOptionsToPackage{hyphens}{url}
%
\documentclass[
  ignorenonframetext,
]{beamer}
\usepackage{pgfpages}
\setbeamertemplate{caption}[numbered]
\setbeamertemplate{caption label separator}{: }
\setbeamercolor{caption name}{fg=normal text.fg}
\beamertemplatenavigationsymbolsempty
% Prevent slide breaks in the middle of a paragraph
\widowpenalties 1 10000
\raggedbottom
\setbeamertemplate{part page}{
  \centering
  \begin{beamercolorbox}[sep=16pt,center]{part title}
    \usebeamerfont{part title}\insertpart\par
  \end{beamercolorbox}
}
\setbeamertemplate{section page}{
  \centering
  \begin{beamercolorbox}[sep=12pt,center]{part title}
    \usebeamerfont{section title}\insertsection\par
  \end{beamercolorbox}
}
\setbeamertemplate{subsection page}{
  \centering
  \begin{beamercolorbox}[sep=8pt,center]{part title}
    \usebeamerfont{subsection title}\insertsubsection\par
  \end{beamercolorbox}
}
\AtBeginPart{
  \frame{\partpage}
}
\AtBeginSection{
  \ifbibliography
  \else
    \frame{\sectionpage}
  \fi
}
\AtBeginSubsection{
  \frame{\subsectionpage}
}
\usepackage{lmodern}
\usepackage{amssymb,amsmath}
\usepackage{ifxetex,ifluatex}
\ifnum 0\ifxetex 1\fi\ifluatex 1\fi=0 % if pdftex
  \usepackage[T1]{fontenc}
  \usepackage[utf8]{inputenc}
  \usepackage{textcomp} % provide euro and other symbols
\else % if luatex or xetex
  \usepackage{unicode-math}
  \defaultfontfeatures{Scale=MatchLowercase}
  \defaultfontfeatures[\rmfamily]{Ligatures=TeX,Scale=1}
\fi
% Use upquote if available, for straight quotes in verbatim environments
\IfFileExists{upquote.sty}{\usepackage{upquote}}{}
\IfFileExists{microtype.sty}{% use microtype if available
  \usepackage[]{microtype}
  \UseMicrotypeSet[protrusion]{basicmath} % disable protrusion for tt fonts
}{}
\makeatletter
\@ifundefined{KOMAClassName}{% if non-KOMA class
  \IfFileExists{parskip.sty}{%
    \usepackage{parskip}
  }{% else
    \setlength{\parindent}{0pt}
    \setlength{\parskip}{6pt plus 2pt minus 1pt}}
}{% if KOMA class
  \KOMAoptions{parskip=half}}
\makeatother
\usepackage{xcolor}
\IfFileExists{xurl.sty}{\usepackage{xurl}}{} % add URL line breaks if available
\IfFileExists{bookmark.sty}{\usepackage{bookmark}}{\usepackage{hyperref}}
\hypersetup{
  pdftitle={PS2 Labor Economics {[}2020{]}},
  pdfauthor={Jonas Veit {[}jveit@mail.uni-mannheim.de{]}},
  hidelinks,
  pdfcreator={LaTeX via pandoc}}
\urlstyle{same} % disable monospaced font for URLs
\newif\ifbibliography
\usepackage{color}
\usepackage{fancyvrb}
\newcommand{\VerbBar}{|}
\newcommand{\VERB}{\Verb[commandchars=\\\{\}]}
\DefineVerbatimEnvironment{Highlighting}{Verbatim}{commandchars=\\\{\}}
% Add ',fontsize=\small' for more characters per line
\usepackage{framed}
\definecolor{shadecolor}{RGB}{248,248,248}
\newenvironment{Shaded}{\begin{snugshade}}{\end{snugshade}}
\newcommand{\AlertTok}[1]{\textcolor[rgb]{0.94,0.16,0.16}{#1}}
\newcommand{\AnnotationTok}[1]{\textcolor[rgb]{0.56,0.35,0.01}{\textbf{\textit{#1}}}}
\newcommand{\AttributeTok}[1]{\textcolor[rgb]{0.77,0.63,0.00}{#1}}
\newcommand{\BaseNTok}[1]{\textcolor[rgb]{0.00,0.00,0.81}{#1}}
\newcommand{\BuiltInTok}[1]{#1}
\newcommand{\CharTok}[1]{\textcolor[rgb]{0.31,0.60,0.02}{#1}}
\newcommand{\CommentTok}[1]{\textcolor[rgb]{0.56,0.35,0.01}{\textit{#1}}}
\newcommand{\CommentVarTok}[1]{\textcolor[rgb]{0.56,0.35,0.01}{\textbf{\textit{#1}}}}
\newcommand{\ConstantTok}[1]{\textcolor[rgb]{0.00,0.00,0.00}{#1}}
\newcommand{\ControlFlowTok}[1]{\textcolor[rgb]{0.13,0.29,0.53}{\textbf{#1}}}
\newcommand{\DataTypeTok}[1]{\textcolor[rgb]{0.13,0.29,0.53}{#1}}
\newcommand{\DecValTok}[1]{\textcolor[rgb]{0.00,0.00,0.81}{#1}}
\newcommand{\DocumentationTok}[1]{\textcolor[rgb]{0.56,0.35,0.01}{\textbf{\textit{#1}}}}
\newcommand{\ErrorTok}[1]{\textcolor[rgb]{0.64,0.00,0.00}{\textbf{#1}}}
\newcommand{\ExtensionTok}[1]{#1}
\newcommand{\FloatTok}[1]{\textcolor[rgb]{0.00,0.00,0.81}{#1}}
\newcommand{\FunctionTok}[1]{\textcolor[rgb]{0.00,0.00,0.00}{#1}}
\newcommand{\ImportTok}[1]{#1}
\newcommand{\InformationTok}[1]{\textcolor[rgb]{0.56,0.35,0.01}{\textbf{\textit{#1}}}}
\newcommand{\KeywordTok}[1]{\textcolor[rgb]{0.13,0.29,0.53}{\textbf{#1}}}
\newcommand{\NormalTok}[1]{#1}
\newcommand{\OperatorTok}[1]{\textcolor[rgb]{0.81,0.36,0.00}{\textbf{#1}}}
\newcommand{\OtherTok}[1]{\textcolor[rgb]{0.56,0.35,0.01}{#1}}
\newcommand{\PreprocessorTok}[1]{\textcolor[rgb]{0.56,0.35,0.01}{\textit{#1}}}
\newcommand{\RegionMarkerTok}[1]{#1}
\newcommand{\SpecialCharTok}[1]{\textcolor[rgb]{0.00,0.00,0.00}{#1}}
\newcommand{\SpecialStringTok}[1]{\textcolor[rgb]{0.31,0.60,0.02}{#1}}
\newcommand{\StringTok}[1]{\textcolor[rgb]{0.31,0.60,0.02}{#1}}
\newcommand{\VariableTok}[1]{\textcolor[rgb]{0.00,0.00,0.00}{#1}}
\newcommand{\VerbatimStringTok}[1]{\textcolor[rgb]{0.31,0.60,0.02}{#1}}
\newcommand{\WarningTok}[1]{\textcolor[rgb]{0.56,0.35,0.01}{\textbf{\textit{#1}}}}
\usepackage{graphicx}
\makeatletter
\def\maxwidth{\ifdim\Gin@nat@width>\linewidth\linewidth\else\Gin@nat@width\fi}
\def\maxheight{\ifdim\Gin@nat@height>\textheight\textheight\else\Gin@nat@height\fi}
\makeatother
% Scale images if necessary, so that they will not overflow the page
% margins by default, and it is still possible to overwrite the defaults
% using explicit options in \includegraphics[width, height, ...]{}
\setkeys{Gin}{width=\maxwidth,height=\maxheight,keepaspectratio}
% Set default figure placement to htbp
\makeatletter
\def\fps@figure{htbp}
\makeatother
\setlength{\emergencystretch}{3em} % prevent overfull lines
\providecommand{\tightlist}{%
  \setlength{\itemsep}{0pt}\setlength{\parskip}{0pt}}
\setcounter{secnumdepth}{-\maxdimen} % remove section numbering

\title{PS2 Labor Economics {[}2020{]}}
\subtitle{R Version}
\author{Jonas Veit
{[}\href{mailto:jveit@mail.uni-mannheim.de}{\nolinkurl{jveit@mail.uni-mannheim.de}}{]}}
\date{}

\begin{document}
\frame{\titlepage}

\begin{frame}
\hfill\break
\hfill\break
\hfill\break

\hfill\break
\hfill\break
\hfill\break
\hfill\break
\includegraphics{/home/jonas/Downloads/Uni_Mannheim_Siegel.svg}\\
~\\
~\\
~\\
~\\
\end{frame}

\begin{frame}[fragile]{Estimating the intensive margin labor supply
elasticity}
\protect\hypertarget{estimating-the-intensive-margin-labor-supply-elasticity}{}
\hfill\break
\hfill\break
\hfill\break
Load helpful packages and the cps dataset

\begin{Shaded}
\begin{Highlighting}[]
\KeywordTok{library}\NormalTok{(tidyverse)}
\KeywordTok{library}\NormalTok{(haven)}

\NormalTok{cps =}\StringTok{ }\KeywordTok{read\_dta}\NormalTok{(}\DataTypeTok{file =} \StringTok{\textquotesingle{}\textasciitilde{}/Documents/Programming/R/Labor Economics [2020]//Data for PS2/cps{-}1984{-}2010.dta\textquotesingle{}}\NormalTok{)}
\end{Highlighting}
\end{Shaded}

\hfill\break
\hfill\break
\hfill\break

\begin{block}{\(a)\)}
\protect\hypertarget{a}{}
\begin{Shaded}
\begin{Highlighting}[]
\KeywordTok{summary}\NormalTok{(cps)}
\end{Highlighting}
\end{Shaded}

\begin{verbatim}
##       year         statefip         metro            age           hrswork    
##  Min.   :1984   Min.   : 1.00   Min.   :0.000   Min.   :16.00   Min.   : 0.0  
##  1st Qu.:1991   1st Qu.:12.00   1st Qu.:2.000   1st Qu.:24.00   1st Qu.: 0.0  
##  Median :1999   Median :27.00   Median :2.000   Median :29.00   Median :37.0  
##  Mean   :1998   Mean   :27.48   Mean   :2.408   Mean   :29.92   Mean   :27.5  
##  3rd Qu.:2005   3rd Qu.:39.00   3rd Qu.:3.000   3rd Qu.:36.00   3rd Qu.:40.0  
##  Max.   :2010   Max.   :56.00   Max.   :4.000   Max.   :44.00   Max.   :99.0  
##                                                                               
##     uhrswork        incwage           nKids             mstat  
##  Min.   : 0.00   Min.   :     0   Min.   : 0.0000   Min.   :1  
##  1st Qu.:20.00   1st Qu.:  2430   1st Qu.: 0.0000   1st Qu.:1  
##  Median :40.00   Median : 12016   Median : 0.0000   Median :1  
##  Mean   :30.98   Mean   : 16715   Mean   : 0.6537   Mean   :1  
##  3rd Qu.:40.00   3rd Qu.: 24000   3rd Qu.: 1.0000   3rd Qu.:1  
##  Max.   :99.00   Max.   :713263   Max.   :12.0000   Max.   :1  
##                                                                
##    avgtax_15k         emp_ind          edu_yrs          hsDrop    
##  Min.   :-0.3804   Min.   :0.0000   Min.   : 1.00   Min.   :0.00  
##  1st Qu.: 0.0000   1st Qu.:0.0000   1st Qu.:12.00   1st Qu.:0.00  
##  Median : 0.0000   Median :1.0000   Median :12.00   Median :0.00  
##  Mean   : 0.1347   Mean   :0.7442   Mean   :12.95   Mean   :0.15  
##  3rd Qu.: 0.2404   3rd Qu.:1.0000   3rd Qu.:14.00   3rd Qu.:0.00  
##  Max.   : 0.7400   Max.   :1.0000   Max.   :20.00   Max.   :1.00  
##  NA's   :1763                                                     
##      hsGrad          bachelor         advanced            hisp       
##  Min.   :0.0000   Min.   :0.0000   Min.   :0.00000   Min.   :0.0000  
##  1st Qu.:0.0000   1st Qu.:0.0000   1st Qu.:0.00000   1st Qu.:0.0000  
##  Median :0.0000   Median :0.0000   Median :0.00000   Median :0.0000  
##  Mean   :0.3531   Mean   :0.1594   Mean   :0.05592   Mean   :0.1552  
##  3rd Qu.:1.0000   3rd Qu.:0.0000   3rd Qu.:0.00000   3rd Qu.:0.0000  
##  Max.   :1.0000   Max.   :1.0000   Max.   :1.00000   Max.   :1.0000  
##                                                                      
##     nonwhite     
##  Min.   :0.0000  
##  1st Qu.:0.0000  
##  Median :0.0000  
##  Mean   :0.2557  
##  3rd Qu.:1.0000  
##  Max.   :1.0000  
## 
\end{verbatim}

\begin{Shaded}
\begin{Highlighting}[]
\KeywordTok{mean}\NormalTok{(cps}\OperatorTok{$}\NormalTok{nKids)}
\end{Highlighting}
\end{Shaded}

\begin{verbatim}
## [1] 0.6536786
\end{verbatim}

\begin{Shaded}
\begin{Highlighting}[]
\KeywordTok{mean}\NormalTok{(cps}\OperatorTok{$}\NormalTok{hisp)}
\end{Highlighting}
\end{Shaded}

\begin{verbatim}
## [1] 0.1551964
\end{verbatim}

The average number of childen is \(0.653\)

The share of hispanic women in this data is \(15.5\)\%\\
~\\
~\\
\end{block}

\begin{block}{\(b)\)}
\protect\hypertarget{b}{}
\begin{Shaded}
\begin{Highlighting}[]
\NormalTok{hourly\_wage =}\StringTok{ }\NormalTok{cps}\OperatorTok{$}\NormalTok{incwage }\OperatorTok{/}\StringTok{ }\NormalTok{(}\DecValTok{52}\OperatorTok{*}\NormalTok{cps}\OperatorTok{$}\NormalTok{uhrswork)}

\KeywordTok{mean}\NormalTok{(hourly\_wage,}\DataTypeTok{na.rm =} \OtherTok{TRUE}\NormalTok{)}
\end{Highlighting}
\end{Shaded}

\begin{verbatim}
## [1] 10.09143
\end{verbatim}

\begin{Shaded}
\begin{Highlighting}[]
\KeywordTok{sd}\NormalTok{(hourly\_wage,}\DataTypeTok{na.rm =} \OtherTok{TRUE}\NormalTok{)}
\end{Highlighting}
\end{Shaded}

\begin{verbatim}
## [1] 13.27755
\end{verbatim}

Since there are NaN's in our hourly wage vector we have to use
\textbf{na.rm = TRUE}\\
~\\
\(mean(hourly\_wage) = 10.091\)\\
~\\
\(sd(houry\_wage) = 13.277\)\\
~\\
There are missing values because the column \(uhrswork\) contains zero
rows.\\
Dividing by zero produces NaN's. You can calculate the amount of zeros
by executing the following command.

\begin{Shaded}
\begin{Highlighting}[]
\KeywordTok{sum}\NormalTok{(cps}\OperatorTok{$}\NormalTok{uhrswork }\OperatorTok{==}\StringTok{ }\DecValTok{0}\NormalTok{)}
\end{Highlighting}
\end{Shaded}

\begin{verbatim}
## [1] 55727
\end{verbatim}

Compare it with the number of NaN's in the \(hourly\_wage\) vector.

\begin{Shaded}
\begin{Highlighting}[]
\KeywordTok{sum}\NormalTok{(hourly\_wage }\OperatorTok{==}\StringTok{ \textquotesingle{}NaN\textquotesingle{}}\NormalTok{)}
\end{Highlighting}
\end{Shaded}

\begin{verbatim}
## [1] 55727
\end{verbatim}

\hfill\break
\hfill\break
\hfill\break
\end{block}

\begin{block}{\(c)\)}
\protect\hypertarget{c}{}
\begin{Shaded}
\begin{Highlighting}[]
\KeywordTok{mean}\NormalTok{(cps}\OperatorTok{$}\NormalTok{emp\_ind)}
\end{Highlighting}
\end{Shaded}

\begin{verbatim}
## [1] 0.7441964
\end{verbatim}

\begin{Shaded}
\begin{Highlighting}[]
\KeywordTok{mean}\NormalTok{(cps}\OperatorTok{$}\NormalTok{emp\_ind[cps}\OperatorTok{$}\NormalTok{nKids }\OperatorTok{==}\StringTok{ }\DecValTok{0}\NormalTok{])}
\end{Highlighting}
\end{Shaded}

\begin{verbatim}
## [1] 0.7859709
\end{verbatim}

\begin{Shaded}
\begin{Highlighting}[]
\KeywordTok{mean}\NormalTok{(cps}\OperatorTok{$}\NormalTok{emp\_ind[cps}\OperatorTok{$}\NormalTok{nKids }\OperatorTok{==}\StringTok{ }\DecValTok{1} \OperatorTok{\&}\StringTok{ }\DecValTok{2}\NormalTok{])}
\end{Highlighting}
\end{Shaded}

\begin{verbatim}
## [1] 0.7178345
\end{verbatim}

The overall rate of labor force participation is or \(74.41\)\%\\
~\\
The rate of labor force participation among women who have no children
is or \(78.59\)\%\\
~\\
The rate for 1 or 2 children is or \(71.78\)\%\\
~\\
The rate of labor force participation among women with 10 children is
very high because of the amount of women who have 10 children.

\begin{Shaded}
\begin{Highlighting}[]
\NormalTok{cps }\OperatorTok{\%\textgreater{}\%}\StringTok{ }
\StringTok{  }\KeywordTok{filter}\NormalTok{(nKids }\OperatorTok{==}\StringTok{ }\DecValTok{10}\NormalTok{) }\OperatorTok{\%\textgreater{}\%}
\StringTok{  }\KeywordTok{select}\NormalTok{(emp\_ind)}
\end{Highlighting}
\end{Shaded}

\begin{verbatim}
## # A tibble: 10 x 1
##    emp_ind
##      <dbl>
##  1       1
##  2       0
##  3       1
##  4       0
##  5       1
##  6       0
##  7       1
##  8       1
##  9       1
## 10       0
\end{verbatim}

we can clearly see that there are just 10 women with 10 children in this
dataset and 6 of them are participating in the labor force i.e
\(60\)\%\\
~\\
~\\
\end{block}

\begin{block}{\(d)\)}
\protect\hypertarget{d}{}
\hfill\break

we want to estimate the model
\[\textrm{uhrswork}_i = \beta_0 + \beta_1 \textrm{wage}_i + \varepsilon_i\]

\begin{Shaded}
\begin{Highlighting}[]
\NormalTok{regression1 =}\StringTok{ }\KeywordTok{lm}\NormalTok{(cps}\OperatorTok{$}\NormalTok{uhrswork }\OperatorTok{\textasciitilde{}}\StringTok{ }\NormalTok{hourly\_wage);regression1}
\end{Highlighting}
\end{Shaded}

\begin{verbatim}
## 
## Call:
## lm(formula = cps$uhrswork ~ hourly_wage)
## 
## Coefficients:
## (Intercept)  hourly_wage  
##    37.30545      0.06117
\end{verbatim}

\begin{Shaded}
\begin{Highlighting}[]
\NormalTok{sum\_reg1 =}\StringTok{ }\KeywordTok{summary}\NormalTok{(regression1);}\KeywordTok{summary}\NormalTok{(regression1)[}\DecValTok{4}\NormalTok{]}
\end{Highlighting}
\end{Shaded}

\begin{verbatim}
## $coefficients
##                Estimate  Std. Error    t value Pr(>|t|)
## (Intercept) 37.30545282 0.024755393 1506.96265        0
## hourly_wage  0.06116711 0.001484384   41.20706        0
\end{verbatim}

The estimate for \(\hat{\beta}_1\) is \(0.06117\)\\
~\\
The standard error for \(\hat{\beta}_1\) is \(0.0014\) and the
t-statistic is \(41.20\)\\
We can see that \(Pr(>|t|)\) is \(0\) that means our p value is 0\\
~\\
According to the estimated \(\hat{\beta}_1\) an one dollar increase in
wage should result in an increase of hours worked of 0.0611\\
~\\
Now estimate the elasticity of hours worked with respect to wage

\begin{Shaded}
\begin{Highlighting}[]
\NormalTok{avg\_hourly\_wage =}\StringTok{ }\KeywordTok{mean}\NormalTok{(hourly\_wage,}\DataTypeTok{na.rm =} \OtherTok{TRUE}\NormalTok{)}
\NormalTok{avg\_uhrswork =}\StringTok{ }\KeywordTok{mean}\NormalTok{(cps}\OperatorTok{$}\NormalTok{uhrswork)}
\NormalTok{regression1}\OperatorTok{$}\NormalTok{coefficients[}\DecValTok{2}\NormalTok{]}\OperatorTok{*}\NormalTok{(avg\_hourly\_wage}\OperatorTok{/}\NormalTok{avg\_uhrswork)}
\end{Highlighting}
\end{Shaded}

\begin{verbatim}
## hourly_wage 
##  0.01992475
\end{verbatim}

The estimated elasticity \(e\) is equal to \(0.0199\)\\
~\\
~\\
\end{block}

\begin{block}{\(e)\)}
\protect\hypertarget{e}{}
\hfill\break
Now we want to estimate the log regression
\[\log(\textrm{uhrswork}_i) = \beta_0 + \beta_1 \log(\textrm{wage}_i) + \varepsilon_i\]

\begin{Shaded}
\begin{Highlighting}[]
\NormalTok{log\_uhrswork =}\StringTok{ }\KeywordTok{log}\NormalTok{(cps}\OperatorTok{$}\NormalTok{uhrswork)}
\NormalTok{log\_wage =}\StringTok{ }\KeywordTok{log}\NormalTok{(hourly\_wage)}
\NormalTok{log\_regression1 =}\StringTok{ }\KeywordTok{lm}\NormalTok{(log\_uhrswork }\OperatorTok{\textasciitilde{}}\StringTok{ }\NormalTok{log\_wage);log\_regression1}
\end{Highlighting}
\end{Shaded}

\begin{verbatim}
## 
## Call:
## lm(formula = log_uhrswork ~ log_wage)
## 
## Coefficients:
## (Intercept)     log_wage  
##     3.42886      0.08245
\end{verbatim}

\begin{Shaded}
\begin{Highlighting}[]
\NormalTok{sum\_logreg1 =}\StringTok{ }\KeywordTok{summary}\NormalTok{(log\_regression1);sum\_logreg1[}\DecValTok{4}\NormalTok{]}
\end{Highlighting}
\end{Shaded}

\begin{verbatim}
## $coefficients
##               Estimate   Std. Error   t value Pr(>|t|)
## (Intercept) 3.42886100 0.0015026129 2281.9324        0
## log_wage    0.08245064 0.0006915774  119.2211        0
\end{verbatim}

The estimate for \(\hat{\beta}_1\) is \(0.08245\)\\
~\\
The standard error for \(\hat{\beta}_1\) is \(0.00069\) and the
t-statistic is \(119.2211\)\\
We can see that \(Pr(>|t|)\) is \(0\) that means our p value is 0\\
~\\
\(e_{reg1} = 0.01992475\) ~vs.~~\(e_{log(reg1)} = 0.08245\)\\
~\\
If the wage increases by \(10\)\% we would estimate the effect on hours
worked to be \(10 \times 0.08245 = 0.8245\)\\
~\\
~\\
\end{block}

\begin{block}{\(f)\)}
\protect\hypertarget{f}{}
\hfill\break
I am combining \(log\_wage\) and \(age\) into one dataframe for easier
usage.\\
\(log\_hours\) contains NaN's that means that we have to use the
command: \textbf{use = `complete.obs'}

\begin{Shaded}
\begin{Highlighting}[]
\NormalTok{cov\_dataframe =}\StringTok{ }\KeywordTok{data.frame}\NormalTok{(log\_wage,cps}\OperatorTok{$}\NormalTok{age)}
\KeywordTok{cov}\NormalTok{(cov\_dataframe,}\DataTypeTok{use =} \StringTok{\textquotesingle{}complete.obs\textquotesingle{}}\NormalTok{)}
\end{Highlighting}
\end{Shaded}

\begin{verbatim}
##           log_wage   cps.age
## log_wage 0.9775908  2.219258
## cps.age  2.2192582 55.105299
\end{verbatim}

\begin{Shaded}
\begin{Highlighting}[]
\FloatTok{0.0053}\OperatorTok{*}\NormalTok{(}\KeywordTok{cov}\NormalTok{(log\_wage,cps}\OperatorTok{$}\NormalTok{age,}\DataTypeTok{use =} \StringTok{\textquotesingle{}complete.obs\textquotesingle{}}\NormalTok{)}\OperatorTok{/}\KeywordTok{var}\NormalTok{(log\_wage,}\DataTypeTok{na.rm =} \OtherTok{TRUE}\NormalTok{))}
\end{Highlighting}
\end{Shaded}

\begin{verbatim}
## [1] 0.01203169
\end{verbatim}

\hfill\break
\[OVB = \beta_{age} \frac{Cov(log\_wage,age)}{\tilde{S}^2_{log\_wage}} = 0.0053 \frac{2.219}{0.997} = 0.012\]\\
We estimate the OVB to be around 0.012 i.e a elasticity change of
0.012\\
~\\
~\\
\end{block}

\begin{block}{\(g)\)}
\protect\hypertarget{g}{}
\hfill\break
\[\hat{\beta} = \beta + \beta_{age} \frac{Cov(log\_wage,age)}{\tilde{S}^2_{log\_wage}} \Leftrightarrow \beta = \hat{\beta} -  \beta_{age} \frac{Cov(age,log\_wage)}{\tilde{S}^2_{log\_wage}} = 0.08245 - 0.012 = 0.07045\]\\
~\\
~\\
\end{block}

\begin{block}{\(h)\)}
\protect\hypertarget{h}{}
\hfill\break
We want to estimate the model\\
\[\log(\textrm{uhrswork}_i) = \beta_0 + \beta_1 \log(\textrm{wage}_i) +\beta_2\textrm{edu_years} + \beta_3\textrm{age} + \beta_4 \textrm{age}^2 + \beta_5 \textrm{year} + \beta_6 \textrm{hisp} + \varepsilon_i\]

\begin{Shaded}
\begin{Highlighting}[]
\NormalTok{age\_squared =}\StringTok{ }\NormalTok{cps}\OperatorTok{$}\NormalTok{age}\OperatorTok{\^{}}\DecValTok{2}
\NormalTok{log\_regression2 =}\StringTok{ }\KeywordTok{lm}\NormalTok{(}\DataTypeTok{data =}\NormalTok{ cps,}\DataTypeTok{formula =}\NormalTok{ log\_uhrswork }\OperatorTok{\textasciitilde{}}\StringTok{ }\NormalTok{log\_wage }\OperatorTok{+}\StringTok{ }\NormalTok{edu\_yrs }\OperatorTok{+}\StringTok{ }\NormalTok{age }\OperatorTok{+}\StringTok{ }\NormalTok{age\_squared }\OperatorTok{+}\StringTok{ }\NormalTok{year }\OperatorTok{+}\StringTok{ }\NormalTok{hisp);log\_regression2}
\end{Highlighting}
\end{Shaded}

\begin{verbatim}
## 
## Call:
## lm(formula = log_uhrswork ~ log_wage + edu_yrs + age + age_squared + 
##     year + hisp, data = cps)
## 
## Coefficients:
## (Intercept)     log_wage      edu_yrs          age  age_squared         year  
##   8.0283921    0.0659372    0.0069349    0.0412828   -0.0005755   -0.0026781  
##        hisp  
##   0.0096460
\end{verbatim}

\hfill\break
The estimated labor supply elasticity does not stay the same. We are now
controlling for other variables. Therefore we are able to adress the
omitted variable bias.\\
~\\
~\\
\end{block}

\begin{block}{\(i)\)}
\protect\hypertarget{i}{}
\hfill\break
Here we create a new DataFrame with all women with no childen which we
access in the \(lm()\) command.\\
*** Moreover we create new vectors for \(\log(wage)\) ,
\(\log(uhrswork)\) and \(\textrm{age}^2\) conditioned on
\(nKids == 0\)\\
Note that I used a different approach to create the DataFrame and
vectors using the package `dplyr' which belongs to the `tidyverse'
package that we loaded at the beginning\\
~\\
We could have also used the following base-R commands:\\
\[\textrm{cps_nochildren = cps \$ [cps \$ nKids == 0,]}\]\\
~\\
\[\textrm{log_wage_nochildren} = \log(\textrm{cps_nochildren \$ incwage / (52*cps_nochildren \$ uhrswork}))\]\\
\[...\]\\
~\\
But using `dplyr' makes the code much easier to read!\\

\begin{Shaded}
\begin{Highlighting}[]
\NormalTok{cps\_nochildren =}\StringTok{ }\NormalTok{cps }\OperatorTok{\%\textgreater{}\%}
\StringTok{  }\KeywordTok{filter}\NormalTok{(nKids }\OperatorTok{==}\StringTok{ }\DecValTok{0}\NormalTok{)}

\NormalTok{log\_wage\_nochildren =}\StringTok{ }\NormalTok{cps }\OperatorTok{\%\textgreater{}\%}
\StringTok{  }\KeywordTok{filter}\NormalTok{(nKids }\OperatorTok{==}\StringTok{ }\DecValTok{0}\NormalTok{) }\OperatorTok{\%\textgreater{}\%}
\StringTok{  }\KeywordTok{select}\NormalTok{(incwage,uhrswork) }\OperatorTok{\%\textgreater{}\%}
\StringTok{  }\KeywordTok{transmute}\NormalTok{(}\KeywordTok{log}\NormalTok{(incwage}\OperatorTok{/}\NormalTok{(}\DecValTok{52}\OperatorTok{*}\NormalTok{uhrswork))) }\OperatorTok{\%\textgreater{}\%}
\StringTok{  }\KeywordTok{pull}\NormalTok{()}


\NormalTok{log\_uhrswork\_nochildren =}\StringTok{ }\NormalTok{cps }\OperatorTok{\%\textgreater{}\%}
\StringTok{  }\KeywordTok{filter}\NormalTok{(nKids }\OperatorTok{==}\StringTok{ }\DecValTok{0}\NormalTok{) }\OperatorTok{\%\textgreater{}\%}
\StringTok{  }\KeywordTok{select}\NormalTok{(uhrswork) }\OperatorTok{\%\textgreater{}\%}
\StringTok{  }\KeywordTok{transmute}\NormalTok{(}\KeywordTok{log}\NormalTok{(uhrswork)) }\OperatorTok{\%\textgreater{}\%}
\StringTok{  }\KeywordTok{pull}\NormalTok{()}

\NormalTok{age\_squared\_nochildren =}\StringTok{ }\NormalTok{cps\_nochildren }\OperatorTok{\%\textgreater{}\%}
\StringTok{  }\KeywordTok{select}\NormalTok{(age) }\OperatorTok{\%\textgreater{}\%}
\StringTok{  }\KeywordTok{transmute}\NormalTok{(age}\OperatorTok{\^{}}\DecValTok{2}\NormalTok{) }\OperatorTok{\%\textgreater{}\%}
\StringTok{  }\KeywordTok{pull}\NormalTok{()}
\end{Highlighting}
\end{Shaded}

\begin{Shaded}
\begin{Highlighting}[]
\NormalTok{log\_regression\_nochildren =}\StringTok{ }\KeywordTok{lm}\NormalTok{(}\DataTypeTok{data =}\NormalTok{ cps\_nochildren,}\DataTypeTok{formula =}\NormalTok{ log\_uhrswork\_nochildren }\OperatorTok{\textasciitilde{}}\StringTok{ }\NormalTok{log\_wage\_nochildren }\OperatorTok{+}\StringTok{ }\NormalTok{edu\_yrs }\OperatorTok{+}\StringTok{ }\NormalTok{age }\OperatorTok{+}\StringTok{ }\NormalTok{age\_squared\_nochildren }\OperatorTok{+}\StringTok{ }\NormalTok{year }\OperatorTok{+}\StringTok{ }\NormalTok{hisp);log\_regression\_nochildren}
\end{Highlighting}
\end{Shaded}

\begin{verbatim}
## 
## Call:
## lm(formula = log_uhrswork_nochildren ~ log_wage_nochildren + 
##     edu_yrs + age + age_squared_nochildren + year + hisp, data = cps_nochildren)
## 
## Coefficients:
##            (Intercept)     log_wage_nochildren                 edu_yrs  
##              7.5427277               0.0609923               0.0057563  
##                    age  age_squared_nochildren                    year  
##              0.0520038              -0.0007289              -0.0025045  
##                   hisp  
##              0.0088666
\end{verbatim}

\hfill\break

\begin{Shaded}
\begin{Highlighting}[]
\NormalTok{cps\_children =}\StringTok{ }\NormalTok{cps }\OperatorTok{\%\textgreater{}\%}
\StringTok{  }\KeywordTok{filter}\NormalTok{(nKids }\OperatorTok{\textgreater{}}\StringTok{ }\DecValTok{1}\NormalTok{)}

\NormalTok{log\_wage\_children =}\StringTok{ }\NormalTok{cps }\OperatorTok{\%\textgreater{}\%}
\StringTok{  }\KeywordTok{filter}\NormalTok{(nKids }\OperatorTok{\textgreater{}}\StringTok{ }\DecValTok{1}\NormalTok{) }\OperatorTok{\%\textgreater{}\%}
\StringTok{  }\KeywordTok{select}\NormalTok{(incwage,uhrswork) }\OperatorTok{\%\textgreater{}\%}
\StringTok{  }\KeywordTok{transmute}\NormalTok{(}\KeywordTok{log}\NormalTok{(incwage}\OperatorTok{/}\NormalTok{(}\DecValTok{52}\OperatorTok{*}\NormalTok{uhrswork))) }\OperatorTok{\%\textgreater{}\%}
\StringTok{  }\KeywordTok{pull}\NormalTok{()}


\NormalTok{log\_uhrswork\_children =}\StringTok{ }\NormalTok{cps }\OperatorTok{\%\textgreater{}\%}
\StringTok{  }\KeywordTok{filter}\NormalTok{(nKids }\OperatorTok{\textgreater{}}\StringTok{ }\DecValTok{1}\NormalTok{) }\OperatorTok{\%\textgreater{}\%}
\StringTok{  }\KeywordTok{select}\NormalTok{(uhrswork) }\OperatorTok{\%\textgreater{}\%}
\StringTok{  }\KeywordTok{transmute}\NormalTok{(}\KeywordTok{log}\NormalTok{(uhrswork)) }\OperatorTok{\%\textgreater{}\%}
\StringTok{  }\KeywordTok{pull}\NormalTok{()}

\NormalTok{age\_squared\_children =}\StringTok{ }\NormalTok{cps\_children }\OperatorTok{\%\textgreater{}\%}
\StringTok{  }\KeywordTok{select}\NormalTok{(age) }\OperatorTok{\%\textgreater{}\%}
\StringTok{  }\KeywordTok{transmute}\NormalTok{(age}\OperatorTok{\^{}}\DecValTok{2}\NormalTok{) }\OperatorTok{\%\textgreater{}\%}
\StringTok{  }\KeywordTok{pull}\NormalTok{()}
\end{Highlighting}
\end{Shaded}

\begin{Shaded}
\begin{Highlighting}[]
\NormalTok{log\_regression\_children =}\StringTok{ }\KeywordTok{lm}\NormalTok{(}\DataTypeTok{data =}\NormalTok{ cps\_children,}\DataTypeTok{formula =}\NormalTok{ log\_uhrswork\_children }\OperatorTok{\textasciitilde{}}\StringTok{ }\NormalTok{log\_wage\_children }\OperatorTok{+}\StringTok{ }\NormalTok{edu\_yrs }\OperatorTok{+}\StringTok{ }\NormalTok{age }\OperatorTok{+}\StringTok{ }\NormalTok{age\_squared\_children }\OperatorTok{+}\StringTok{ }\NormalTok{year }\OperatorTok{+}\StringTok{ }\NormalTok{hisp);log\_regression\_children}
\end{Highlighting}
\end{Shaded}

\begin{verbatim}
## 
## Call:
## lm(formula = log_uhrswork_children ~ log_wage_children + edu_yrs + 
##     age + age_squared_children + year + hisp, data = cps_children)
## 
## Coefficients:
##          (Intercept)     log_wage_children               edu_yrs  
##            6.6743797             0.0682655             0.0042289  
##                  age  age_squared_children                  year  
##            0.0152948            -0.0001984            -0.0017839  
##                 hisp  
##            0.0142629
\end{verbatim}

\hfill\break
\hfill\break
\hfill\break
\end{block}

\begin{block}{\(j)\)}
\protect\hypertarget{j}{}
\hfill\break

\hfill\break
\hfill\break
\hfill\break
\hfill\break
\hfill\break
\end{block}
\end{frame}

\begin{frame}[fragile]{Minimum Wage Application}
\protect\hypertarget{minimum-wage-application}{}
\hfill\break
\hfill\break
\hfill\break
Load the corresponding dataset.

\begin{Shaded}
\begin{Highlighting}[]
\NormalTok{dubelesterreich\_minwage =}\StringTok{ }\KeywordTok{read\_dta}\NormalTok{(}\DataTypeTok{file =} \StringTok{"\textasciitilde{}/Documents/Programming/R/Labor Economics [2020]/Data for PS2/dubelesterreich\_minwage.dta"}\NormalTok{)}

\NormalTok{dubelesterreich\_empdata =}\StringTok{ }\KeywordTok{read\_dta}\NormalTok{(}\DataTypeTok{file =} \StringTok{"\textasciitilde{}/Documents/Programming/R/Labor Economics [2020]/Data for PS2/dubelesterreich\_empdata.dta"}\NormalTok{)}

\NormalTok{dubelesterreich\_empdata\_contig\_minwage =}\StringTok{ }\KeywordTok{read\_dta}\NormalTok{(}\DataTypeTok{file =} \StringTok{"\textasciitilde{}/Documents/Programming/R/Labor Economics [2020]/Data for PS2/dubelesterreich\_empdata\_contig\_minwage.dta"}\NormalTok{)}
\end{Highlighting}
\end{Shaded}

\hfill\break
\hfill\break
\hfill\break

\begin{block}{\(1)\) Prepare the dataset}
\protect\hypertarget{prepare-the-dataset}{}
\hfill\break
Create the meanwage vector and append it to the dataframe.\\

\begin{Shaded}
\begin{Highlighting}[]
\NormalTok{minwage =}\StringTok{ }\KeywordTok{rep}\NormalTok{(}\DecValTok{0}\NormalTok{,}\KeywordTok{length}\NormalTok{(dubelesterreich\_minwage}\OperatorTok{$}\NormalTok{st\_mw))}

\ControlFlowTok{for}\NormalTok{ (i }\ControlFlowTok{in} \DecValTok{1}\OperatorTok{:}\KeywordTok{length}\NormalTok{(}\KeywordTok{t}\NormalTok{(minwage)))\{}
  \ControlFlowTok{if}\NormalTok{ (}\KeywordTok{is.na}\NormalTok{(dubelesterreich\_minwage}\OperatorTok{$}\NormalTok{st\_mw[i] }\OperatorTok{==}\StringTok{ }\OtherTok{TRUE}\NormalTok{))\{}
\NormalTok{    minwage[i] =}\StringTok{ }\NormalTok{dubelesterreich\_minwage}\OperatorTok{$}\NormalTok{fed\_mw[i] }
\NormalTok{  \} }\ControlFlowTok{else} \ControlFlowTok{if}\NormalTok{ (dubelesterreich\_minwage}\OperatorTok{$}\NormalTok{st\_mw[i] }\OperatorTok{\textgreater{}}\StringTok{ }\NormalTok{dubelesterreich\_minwage}\OperatorTok{$}\NormalTok{fed\_mw[i])\{}
\NormalTok{    minwage[i] =}\StringTok{ }\NormalTok{dubelesterreich\_minwage}\OperatorTok{$}\NormalTok{st\_mw[i]}
\NormalTok{  \} }\ControlFlowTok{else} \ControlFlowTok{if}\NormalTok{ (dubelesterreich\_minwage}\OperatorTok{$}\NormalTok{st\_mw[i] }\OperatorTok{==}\StringTok{ }\NormalTok{dubelesterreich\_minwage}\OperatorTok{$}\NormalTok{fed\_mw[i])\{}
\NormalTok{    minwage[i] =}\StringTok{ }\NormalTok{dubelesterreich\_minwage}\OperatorTok{$}\NormalTok{st\_mw[i]}
\NormalTok{  \} }\ControlFlowTok{else}\NormalTok{ \{}
\NormalTok{    minwage[i] =}\StringTok{ }\NormalTok{dubelesterreich\_minwage}\OperatorTok{$}\NormalTok{fed\_mw[i]}
\NormalTok{  \}}
\NormalTok{\}}

\NormalTok{dubelesterreich\_minwage =}\StringTok{ }\NormalTok{dubelesterreich\_minwage }\OperatorTok{\%\textgreater{}\%}
\StringTok{  }\KeywordTok{add\_column}\NormalTok{(minwage)}
\end{Highlighting}
\end{Shaded}

\hfill\break

\begin{Shaded}
\begin{Highlighting}[]
\KeywordTok{summary}\NormalTok{(dubelesterreich\_minwage}\OperatorTok{$}\NormalTok{st\_mw)}
\end{Highlighting}
\end{Shaded}

\begin{verbatim}
##    Min. 1st Qu.  Median    Mean 3rd Qu.    Max.    NA's 
##   3.350   4.650   5.750   5.579   6.500   7.930    3904
\end{verbatim}

\begin{Shaded}
\begin{Highlighting}[]
\KeywordTok{summary}\NormalTok{(dubelesterreich\_minwage}\OperatorTok{$}\NormalTok{fed\_mw)}
\end{Highlighting}
\end{Shaded}

\begin{verbatim}
##    Min. 1st Qu.  Median    Mean 3rd Qu.    Max. 
##   3.350   3.350   4.250   4.421   5.150   5.850
\end{verbatim}

\begin{Shaded}
\begin{Highlighting}[]
\KeywordTok{summary}\NormalTok{(minwage)}
\end{Highlighting}
\end{Shaded}

\begin{verbatim}
##    Min. 1st Qu.  Median    Mean 3rd Qu.    Max. 
##   3.350   3.800   4.450   4.602   5.150   7.930
\end{verbatim}

\hfill\break
\(a) \  \ \ \textrm{&} \ \ \ b)\)\\
\(\min = 3.350\)\\
\(\max = 7.930\)\\
~\\

\begin{Shaded}
\begin{Highlighting}[]
\KeywordTok{write\_dta}\NormalTok{(}\KeywordTok{as.data.frame}\NormalTok{(minwage),}\DataTypeTok{path =} \StringTok{"\textasciitilde{}/Documents/Programming/R/Labor Economics [2020]/minwage\_clean.dta"}\NormalTok{)}
\end{Highlighting}
\end{Shaded}

\hfill\break
\hfill\break
Save the minwage vector as a file.

\begin{Shaded}
\begin{Highlighting}[]
\KeywordTok{write\_dta}\NormalTok{(}\KeywordTok{as.data.frame}\NormalTok{(minwage),}\DataTypeTok{path =} \StringTok{"\textasciitilde{}/Documents/Programming/R/Labor Economics [2020]/minwage\_clean.dta"}\NormalTok{)}
\end{Highlighting}
\end{Shaded}

\hfill\break
\hfill\break
\(c)\)\\

\begin{Shaded}
\begin{Highlighting}[]
\KeywordTok{ggplot}\NormalTok{(dubelesterreich\_minwage,}\KeywordTok{aes}\NormalTok{(}\DataTypeTok{color =}\NormalTok{ year)) }\OperatorTok{+}
\StringTok{  }\KeywordTok{geom\_line}\NormalTok{(}\KeywordTok{aes}\NormalTok{(}\DataTypeTok{x =}\NormalTok{ year , }\DataTypeTok{y =}\NormalTok{ minwage))}
\end{Highlighting}
\end{Shaded}

\includegraphics{PS2_2_files/figure-beamer/unnamed-chunk-23-1.pdf}

\hfill\break
\hfill\break
\hfill\break
\$\$
\end{block}
\end{frame}

\begin{frame}[fragile]{Estimating the Effect of Minimum Wages}
\protect\hypertarget{estimating-the-effect-of-minimum-wages}{}
\hfill\break
\hfill\break
Load the corresponding dataset.\\
I have also removed all other variables created for Part 1 of this
Problem Set to have a cleaner workspace.\\

\begin{Shaded}
\begin{Highlighting}[]
\KeywordTok{rm}\NormalTok{(}\DataTypeTok{list =} \KeywordTok{ls}\NormalTok{())}

\NormalTok{empdata\_minwage =}\StringTok{ }\KeywordTok{read\_dta}\NormalTok{(}\DataTypeTok{file =} \StringTok{"\textasciitilde{}/Documents/Programming/R/Labor Economics [2020]/Data for PS2/dubelesterreich\_empdata\_contig\_minwage.dta"}\NormalTok{)}
\end{Highlighting}
\end{Shaded}

\hfill\break
\hfill\break
\hfill\break

\begin{block}{\(a)\)}
\protect\hypertarget{a-1}{}
\hfill\break
\hfill\break
Equation \((1)\) in the Dube, Lester and Reich article is the
following\\
~\\
\[\ln_{it} = \alpha + \eta\ln(MW_{it}) + \delta\ln(y_{it})^{TOT} + \gamma\ln(pop_{it}) + \phi_i + \tau_t + \varepsilon_{it}\]\\
~\\
~\\
First we want to estimate\\
~\\

\[\log(\textrm{earnings_rest}) = \alpha + \eta\log(\textrm{minwage}) + \delta\log(\textrm{earnings_tot}) + \varepsilon_{it}\]\\

\begin{Shaded}
\begin{Highlighting}[]
\NormalTok{log\_regression1 =}\StringTok{ }\KeywordTok{lm}\NormalTok{(}\DataTypeTok{data =}\NormalTok{ empdata\_minwage, }\DataTypeTok{formula =}\NormalTok{ logearnings\_rest }\OperatorTok{\textasciitilde{}}\StringTok{ }\NormalTok{logminwage }\OperatorTok{+}\StringTok{ }\NormalTok{logearnings\_tot);log\_regression1}
\end{Highlighting}
\end{Shaded}

\begin{verbatim}
## 
## Call:
## lm(formula = logearnings_rest ~ logminwage + logearnings_tot, 
##     data = empdata_minwage)
## 
## Coefficients:
##     (Intercept)       logminwage  logearnings_tot  
##          0.9147           0.5042           0.5542
\end{verbatim}

\begin{Shaded}
\begin{Highlighting}[]
\KeywordTok{summary}\NormalTok{(log\_regression1)[}\DecValTok{4}\NormalTok{]}
\end{Highlighting}
\end{Shaded}

\begin{verbatim}
## $coefficients
##                  Estimate  Std. Error   t value Pr(>|t|)
## (Intercept)     0.9147217 0.013707810  66.72997        0
## logminwage      0.5041512 0.005336642  94.46974        0
## logearnings_tot 0.5542416 0.002657709 208.54109        0
\end{verbatim}

\hfill\break
\(\hat{\eta} = 0.5042\)\\
~\\
The coefficient is statistically significant with a t-value of \(94.46\)
and a p-value of \(0\)\\
~\\
~\\
\end{block}

\begin{block}{\(b)\)}
\protect\hypertarget{b-1}{}
\hfill\break

\begin{Shaded}
\begin{Highlighting}[]
\KeywordTok{attach}\NormalTok{(empdata\_minwage)}

\NormalTok{period =}\StringTok{ }\NormalTok{empdata\_minwage }\OperatorTok{\%\textgreater{}\%}
\StringTok{  }\KeywordTok{select}\NormalTok{(period)}

\NormalTok{dummy\_dataframe =}\StringTok{ }\KeywordTok{as.data.frame}\NormalTok{(}\KeywordTok{matrix}\NormalTok{(}\DataTypeTok{nrow =} \KeywordTok{length}\NormalTok{(}\KeywordTok{t}\NormalTok{(period)),}\DataTypeTok{ncol =} \KeywordTok{length}\NormalTok{(}\KeywordTok{min}\NormalTok{(period)}\OperatorTok{:}\KeywordTok{max}\NormalTok{(period))))}
\KeywordTok{colnames}\NormalTok{(dummy\_dataframe) =}\StringTok{ }\KeywordTok{paste}\NormalTok{(}\StringTok{"period"}\NormalTok{,}\DecValTok{1}\OperatorTok{:}\KeywordTok{length}\NormalTok{(dummy\_dataframe),}\DataTypeTok{sep =} \StringTok{""}\NormalTok{)}


\ControlFlowTok{for}\NormalTok{ (i }\ControlFlowTok{in} \DecValTok{1}\OperatorTok{:}\KeywordTok{length}\NormalTok{(}\KeywordTok{min}\NormalTok{(period)}\OperatorTok{:}\KeywordTok{max}\NormalTok{(period)))\{}
\NormalTok{   k =}\StringTok{ }\NormalTok{(i }\OperatorTok{+}\StringTok{ }\KeywordTok{min}\NormalTok{(period) }\DecValTok{{-}1}\NormalTok{)}
\NormalTok{   period\_dummy =}\StringTok{ }\KeywordTok{paste}\NormalTok{(}\StringTok{"period"}\NormalTok{,i,}\DataTypeTok{sep =} \StringTok{""}\NormalTok{)}
\NormalTok{   dummy\_dataframe[,i] =}\StringTok{ }\NormalTok{(}\KeywordTok{assign}\NormalTok{(period\_dummy,}\KeywordTok{as.numeric}\NormalTok{(period }\OperatorTok{==}\StringTok{ }\NormalTok{k))) }
\NormalTok{\}}



\KeywordTok{lm}\NormalTok{(}\DataTypeTok{data =}\NormalTok{ dummy\_dataframe,}\DataTypeTok{formula =}\NormalTok{ logearnings\_rest }\OperatorTok{\textasciitilde{}}\StringTok{ }\NormalTok{logminwage }\OperatorTok{+}\StringTok{ }\NormalTok{logearnings\_tot }\OperatorTok{+}\StringTok{ }\NormalTok{.)}
\end{Highlighting}
\end{Shaded}

\begin{verbatim}
## 
## Call:
## lm(formula = logearnings_rest ~ logminwage + logearnings_tot + 
##     ., data = dummy_dataframe)
## 
## Coefficients:
##     (Intercept)       logminwage  logearnings_tot          period1  
##       1.0461973        0.4841454        0.5418954       -0.0117022  
##         period2          period3          period4          period5  
##      -0.0379120        0.0096442       -0.0368544       -0.0439516  
##         period6          period7          period8          period9  
##      -0.0540828       -0.0011221       -0.0528744       -0.0608269  
##        period10         period11         period12         period13  
##      -0.0423389       -0.0021580       -0.0538928       -0.0588076  
##        period14         period15         period16         period17  
##      -0.0359589       -0.0037129       -0.0380129       -0.0734469  
##        period18         period19         period20         period21  
##      -0.0261712        0.0134041       -0.0300735       -0.0493812  
##        period22         period23         period24         period25  
##      -0.0209259        0.0194947       -0.0289022       -0.0468806  
##        period26         period27         period28         period29  
##      -0.0159506        0.0288481       -0.0620302       -0.0776660  
##        period30         period31         period32         period33  
##      -0.0376042       -0.0268723       -0.0768065       -0.0784154  
##        period34         period35         period36         period37  
##      -0.0475149        0.0032291       -0.0484005       -0.0719876  
##        period38         period39         period40         period41  
##      -0.0333297        0.0035027       -0.0376056       -0.0602525  
##        period42         period43         period44         period45  
##      -0.0162025        0.0110575       -0.0307410       -0.0463679  
##        period46         period47         period48         period49  
##      -0.0140454        0.0217719       -0.0256609       -0.0381070  
##        period50         period51         period52         period53  
##      -0.0038756        0.0299994       -0.0154519       -0.0351850  
##        period54         period55         period56         period57  
##      -0.0006799        0.0321498       -0.0161864       -0.0184508  
##        period58         period59         period60         period61  
##       0.0016057        0.0299770       -0.0110195       -0.0350209  
##        period62         period63         period64         period65  
##       0.0029829        0.0333125       -0.0161822       -0.0275188  
##        period66  
##              NA
\end{verbatim}
\end{block}
\end{frame}

\end{document}
